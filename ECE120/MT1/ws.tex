\documentclass{article}
\usepackage{graphicx} % Required for inserting images
\usepackage{amsmath}
\usepackage{amssymb}
\usepackage{enumitem}
\usepackage{subfig}

\begin{document}
\begin{center}
\textbf{
{\Large HKN ECE 120 Midterm 1 Worksheet}
}
\end{center} 
\noindent\makebox[\linewidth]{\rule{\linewidth}{0.2pt}}


\section*{Binary Representations}
\subsection*{Problem 1}
Write these conversions in decimal. Truncate if necessary.
\begin{enumerate}[label=\alph*.]
    \item Convert $100101_2$ to a 6-bit unsigned integer.
    \item Convert $100101_2$ to a 6-bit signed magnitude integer.
    \item Convert $100101_2$ to a 6-bit 2's complement integer.
    \item Convert $0 1110 1110_2$ to a 9-bit unsigned integer.
    \item Convert $0 1110 1110_2$ to a 9-bit 2's complement integer.
    \item Convert $1001 0010 1101_2$ to a 11-bit unsigned integer.
    \item Convert $1001 0010 1101_2$ to a 9-bit 2's complement integer.
    \item Convert $0010 1101_2$ to a 12-bit unsigned integer.
    \item Convert $1 0111_2$ to a 16-bit signed integer. 
\end{enumerate}

\subsection*{Problem 2}
Write these conversions in binary. Truncate if necessary.
\begin{enumerate}[label=\alph*.]
    \item Convert $51_{10}$ to a 8-bit unsigned integer.
    \item Convert $51_{10}$ to a 8-bit signed magnitude integer.
    \item Convert $51_{10}$ to a 8-bit 2's complement integer.
    \item Convert $-240_{10}$ to a 9-bit signed magnitude integer.
    \item Convert $-240_{10}$ to a 9-bit 2's complement integer.
    \item Convert $1171_{10}$ to a 11-bit unsigned integer.
    \item Convert $1171_{10}$ to a 11-bit 2's complement integer.
    \item Convert $65_{10}$ to a 12-bit unsigned integer.
    \item Convert $-23309_{10}$ to a 16-bit 2's complement integer. 
\end{enumerate}

\section*{Other Representations}
\subsection*{Problem 1}
Convert these binary values to hexadecimal.
\begin{enumerate}[label=\alph*.]
    \item $0010 1011 0101 0110$
    \item $1001 0100 1000 1111$
    \item $0011 1100 0001 0010$
    \item $1011 1110 1110 1111$
    \item $1111 0000 0000 1101$
\end{enumerate}

\subsection*{Problem 2}
Convert these hexadecimal values to binary.
\begin{enumerate}[label=\alph*.]
    \item x37A5
    \item x2009
    \item x1F06
    \item x2FFE
    \item xDEADBEEF
\end{enumerate}

\subsection*{Problem 3}
Convert these hexadecimal values to ASCII.
\begin{enumerate}[label=\alph*.]
    \item x4A
    \item x2F
    \item x0D
    \item x4045
    \item x6E6F
\end{enumerate}

\subsection*{Problem 4}
Convert these ASCII characters to binary.
\begin{enumerate}[label=\alph*.]
    \item 'i'
    \item '#'
    \item 'M'
    \item '!'
    \item "bob"
\end{enumerate}

\subsection*{Problem 5}
True or False?
\begin{enumerate}[label=\alph*.]
    \item An integer with 11 hexadecimal values is at most a 88-bit integer.
    \item The shortest hexadecimal string that we can encode any 69-bit unsigned integer into is 18 characters long.
    \item All uppercase letters in ASCII start with the binary string $0100$.
    \item All lowercase letters in ASCII start with the binary string $011$.
    \item There is an ASCII character that directly corresponds to x8A.
    \item ASCII characters are usually stored as signed 8-bit integers.
    \item The control characters in ASCII were originally used as special codes for teletypes, keyboards used for electrical telegraphs. 
\end{enumerate}

\newpage
\section*{Binary Operations} % include bitmasks

\subsection*{Problem 1}
Perform the following operations. 
\begin{enumerate}[label=\alph*.]
    \item $1_2$ AND $0_2$
    \item $1_2$ OR $0_2$
    \item $10010010_2$ AND $01111011_2$
    \item $001010_2$ OR $111101_2$
    \item x8618 AND x7507
    \item $1_2$ XOR $1_2$
    \item xCA09 XOR x0990
    \item NOT $1001110100110101_2$
    \item $1001001101_2$ NAND $0110101110_2$
    \item $100011_2$ NOR $001000_2$
    \item x908 XNOR xA51
\end{enumerate}

\subsection*{Problem 2}
Perform the following operations on unsigned integers. Assume the number of bits given. Indicate when there is an overflow for operations that have it.
\begin{enumerate}[label=\alph*.]
    \item $100100_2 + 010101_2$
    % \item $11011010_2 - 011010110_2$
    % \item $1001_2 - 1010_2$
    \item $011101_2 + 111011_2$
    \item $1111000_2 \ll 2$ 
    \item $1111000_2 \gg 2$
    \item $000100_2 \gg 2$
\end{enumerate}

\subsection*{Problem 3}
Perform the following operations on signed integers. Assume the number of bits given. Indicate when there is an overflow for operations that have it.
\begin{enumerate}[label=\alph*.]
    \item $110010_2 + 110001_2$
    \item $11011010_2 + 011010110_2$
    \item $1001_2 - 1010_2$
    \item $011101_2 - 111011_2$
    \item $1111000_2 \ll 2$ 
    \item $1111000_2 \gg 2$
    \item $000100_2 \gg 2$
\end{enumerate}

% \subsection*{Problem 4}
% Answer the following questions about bitmasks.
% \begin{enumerate}[label=\alph*.]
%     \item Suppose you have a 6-bit unsigned integer. What does applying AND $110000_2$ return? What does it indicate? 
%     \item Suppose you have a 8-bit signed integer. What does applying AND $10000000_2$ return? What does it indicate? \\ \\ \\
%     Suppose you have a 6-bit unsigned integer that represents 6 lights (1 = on, 0 = off).
%     \item What operation and what mask should we use to enable a single light?
%     \item What operation and what mask should we use to disable a single light?
%     \item What operation and what mask should we use to toggle a single light?
%     \item What operation can we use on these masks to form a new mask if we wanted to toggle more than one light?
% \end{enumerate}

% Not on MT1 last time
% \newpage
% \section*{K-maps and Optimization} % include area, delay heuristic
% \subsection*{Problem 1}
% Find the minimal SOP and POS expressions for the following table.
% \begin{table}[!h]
% \begin{tabular}{|l|l|l|l|}
% \hline
% A & B & C & S \\ \hline
% 0 & 0 & 0 & 1 \\ \hline
% 0 & 0 & 1 & 1 \\ \hline
% 0 & 1 & 0 & 0 \\ \hline
% 0 & 1 & 1 & 0 \\ \hline
% 1 & 0 & 0 & 0 \\ \hline
% 1 & 0 & 1 & 1 \\ \hline
% 1 & 1 & 0 & 1 \\ \hline
% 1 & 1 & 1 & 1 \\ \hline
% \end{tabular}
% \end{table}

% \subsection*{Problem 2}
% Find the minimal SOP and POS expressions for the following table.
% \begin{table}[!h]
% \begin{tabular}{|l|l|l|l|}
% \hline
% A & B & C & S \\ \hline
% 0 & 0 & 0 & 0 \\ \hline
% 0 & 0 & 1 & 1 \\ \hline
% 0 & 1 & 0 & X \\ \hline
% 0 & 1 & 1 & X \\ \hline
% 1 & 0 & 0 & 1 \\ \hline
% 1 & 0 & 1 & 0 \\ \hline
% 1 & 1 & 0 & X \\ \hline
% 1 & 1 & 1 & X \\ \hline
% \end{tabular}
% \end{table}

% \newpage
% \subsection*{Problem 3}
% Find the minimal SOP and POS expressions for the following table.
% \begin{table}[!h]
% \begin{tabular}{|l|l|l|l|l|}
% \hline
% A & B & C & D & S \\ \hline
% 0 & 0 & 0 & 0 & 0 \\ \hline
% 0 & 0 & 0 & 1 & 1 \\ \hline
% 0 & 0 & 1 & 0 & 1 \\ \hline
% 0 & 0 & 1 & 1 & 0 \\ \hline
% 0 & 1 & 0 & 0 & 1 \\ \hline
% 0 & 1 & 0 & 1 & 1 \\ \hline
% 0 & 1 & 1 & 0 & 0 \\ \hline
% 0 & 1 & 1 & 1 & 1 \\ \hline
% 1 & 0 & 0 & 0 & 1 \\ \hline
% 1 & 0 & 0 & 1 & 0 \\ \hline
% 1 & 0 & 1 & 0 & 0 \\ \hline
% 1 & 0 & 1 & 1 & 0 \\ \hline
% 1 & 1 & 0 & 0 & 0 \\ \hline
% 1 & 1 & 0 & 1 & 0 \\ \hline
% 1 & 1 & 1 & 0 & 0 \\ \hline
% 1 & 1 & 1 & 1 & 0 \\ \hline
% \end{tabular}
% \end{table}

% \subsection*{Problem 4}
% Find the minimal SOP and POS expressions for the following table.
% \begin{table}[!h]
% \begin{tabular}{|l|l|l|l|l|}
% \hline
% A & B & C & D & S \\ \hline
% 0 & 0 & 0 & 0 & X \\ \hline
% 0 & 0 & 0 & 1 & 1 \\ \hline
% 0 & 0 & 1 & 0 & 0 \\ \hline
% 0 & 0 & 1 & 1 & 0 \\ \hline
% 0 & 1 & 0 & 0 & X \\ \hline
% 0 & 1 & 0 & 1 & X \\ \hline
% 0 & 1 & 1 & 0 & 1 \\ \hline
% 0 & 1 & 1 & 1 & 1 \\ \hline
% 1 & 0 & 0 & 0 & 1 \\ \hline
% 1 & 0 & 0 & 1 & 0 \\ \hline
% 1 & 0 & 1 & 0 & 0 \\ \hline
% 1 & 0 & 1 & 1 & 0 \\ \hline
% 1 & 1 & 0 & 0 & X \\ \hline
% 1 & 1 & 0 & 1 & X \\ \hline
% 1 & 1 & 1 & 0 & X \\ \hline
% 1 & 1 & 1 & 1 & X \\ \hline
% \end{tabular}
% \end{table}
% \newpage 
% \subsection*{Problem 5}
% Find the area and delay heuristics for the following expressions. Do not include NOT gates.  
% \begin{enumerate}[label=\alph*.]
%     \item $ABC + A'B'C + C'$
%     \item $A + B + C + D(A + B)$
%     \item $ABCDEFGHIJKLMNOPQRSTUVWXY + Z$
%     \item $(AB)'(A+B)'(CD)$
% \end{enumerate}

% \subsection*{Problem 6}
% Implement the following expressions using AND and OR gates, then using NAND and NOR gates only. 
% \begin{enumerate}[label=\alph*.]
%     \item $AB + C$
%     \item $A'B + AB' + ABC' + ABD'$
%     \item $(A+B+C')(A'+B+C)(A+B'+C)$
%     \item $(A+D)(B'+C'+A)$
% \end{enumerate}


\newpage
\section*{IEEE 754 Floating Point}
\subsection*{Problem 1}
Convert the following decimal representations to IEEE-754 floating point.
\begin{enumerate}[label=\alph*.]
    \item 3.625
    \item -18.5
    \item 42.3125
\end{enumerate}

\subsection*{Problem 2}
Convert the following IEEE-754 floating point representations to decimal.
\begin{enumerate}[label=\alph*.]
    \item 0 10000001 11100100000000000000000
    \item 0 10000011 00001000000000000000000
    \item 1 10000011 10010100000000000000000
\end{enumerate}

\section*{C Basics} 
\subsection*{Problem 1}
    Declare the following variables:
        \begin{enumerate}[label=\alph*.]
            \item The signed integer -10 named $x$.
            \item The character `p' named $P$.
            \item The decimal 0.536 as a float named $y$.
            \item The unsigned integer 235 named $ux$.
            \item The decimal 0.46668 as a double named $dy$.
        \end{enumerate}
\subsection*{Problem 2}
    Evaluate the following expressions in C. Assume that the variable a has been declared as 0xECEB and b has been declared as 0x2345.
        \begin{enumerate}[label=\alph*.]
            \item a \& b
            \item a \^{} b
            \item $\sim$ a
            \item a $\mid$ b
        \end{enumerate}

\section*{C Programming}
\subsection*{Problem 1}
    Write code in C for the following tasks. Assume that age is already initialized to 0 and is of type int.
        \begin{enumerate}[label=\alph*.]
            \item Print a prompt message asking the user to input their age.
            \item Store the input in the variable age. 
            \item Print twice of the age you received as an input to the console.
        \end{enumerate}
        
\subsection*{Problem 2}
    Consider the following C code.
    \begin{verbatim}
        int main() {
            for (int i = 0; i < 10; i ++) {
                printf("%d\n", i);

                if (i == 10) {
                    printf("Now i is 10.");
                }
            }
            return 0;
        }
    \end{verbatim}

    \begin{enumerate}[label=\alph*.]
        \item How many times does the program print to the console?
        \item What is the output of this program?
    \end{enumerate}

\subsection*{Problem 3}
    What does the following C code print?

    \begin{verbatim}
    int main() {
        int x = 10;
        if (x = 5) {
            printf("x is 5.");
        } else {
            printf("x is not 5.");
        }
        return 0;
    }
    \end{verbatim}

\subsection*{Problem 4}
    What does the following C code print?
    
    \begin{verbatim}
        int main() {
            int i = 90;
            while (i >= 3) {
                printf("%d ", i);
                i = i/3;
            }
            return 0;
        }
    \end{verbatim}


\end{document}

