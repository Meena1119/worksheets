\documentclass{article} 
 
\usepackage[utf8]{inputenc} 
\usepackage[T1]{fontenc} 
\usepackage{geometry} 
\geometry{a4paper} 
\usepackage{graphicx}
\usepackage{hyperref}
\usepackage{multirow}
 
\title{HKN Latex Reference Guide}
\date{}

\begin{document}
\maketitle

\section*{Starting a Latex File}
\par To start a basic Latex file,
\begin{enumerate}
    \item Include \textbackslash documentclass\{article\} at the beginning of the document
    \item A title can be created by specifying the title name with \textbackslash \{\textit{your\_title}\}. Then include \textbackslash maketitle in the main body of your document.
    \item Start the main body of your document with \textbackslash begin\{document\}. Also specify the end of the document with \textbackslash end\{document\}. This is where you will do most of your writing in Latex.
\end{enumerate}

\section*{Basic Formatting}

\begin{itemize}
\item Sections can be inserted with \textbackslash section\{\textit{your\_heading}\} for a numbered heading and \\ \textbackslash section*\{\textit{your\_heading}\} for an unnumbered heading.
\item Newlines can be inserted with \textbackslash\textbackslash
\item Paragraphs can be inserted with \textbackslash par
\item Pagebreaks can be inserted with \textbackslash newpage
\item Italicize with \textbackslash textit\{\textit{your\_text}\}
\item Bold with \textbackslash textbf\{\textit{your\_text}\}
\item Underline with \textbackslash underline\{\textit{your\_text}\}
\item Center by enclosing what you wish to be centered with \textbackslash begin\{center\} and \textbackslash end\{center\}
%\item Align left with \textbackslash raggedleft
%\item Align right with \textbackslash raggedright
%add indentation%
\end{itemize}

\section*{Math}
\par To insert mathematical expressions into your Latex document, enclose the expression in dollar signs. \\
\par For example:
The function f of x equals x squared minus 5 is: \\
\begin{center}
\$f(x)=x\^\{2\} -5\$ \\
$f(x)=x^{2}-5$
\end{center}
\par There are many different latex codes for mathematical expressions. Generally, one can search online for the math symbol you need.

\section*{Figures}
\begin{enumerate}
    \item To insert figures, include \textbackslash usepackage\{graphicx\} at the beginning of your Latex document.
    \item Also include \textbackslash graphicspath\{ \{\textit{./file\_path/}\} \} to specify the folder that inlcudes the images you want to insert.
    \item Simply use \textbackslash includegraphics\{\textit{file\_name}\} to insert an image. The file extension is not required to be included.
\end{enumerate}
\par See the different examples of figure position available in Latex here: \\
\url{https://www.overleaf.com/project/new/template/19663?id=66518064&templateName=Example+of+positioning+tables+and+figures&latexEngine=pdflatex&texImage=texlive-full%3A2020.1&mainFile=}

\section*{Lists}
\begin{itemize}
    \item Unordered lists can be created with \textbackslash begin\{itemize\} and ended with \textbackslash end\{itemize\}.
    \item Ordered lists can be created with \textbackslash begin\{enumerate\} and ended with \textbackslash end\{enumerate\}.
    \item Individual items in lists are created with \textbackslash item \textit{Your text here}
\end{itemize}

\section*{Tables}
\par To include tables in your document,
\begin{enumerate}
    \item Include \textbackslash usepackage\{tabularx\} at the beginning of your file.
    \item Begin a table with \textbackslash begin\{tabular\}
    \item Specify the format of your table with \{| c | c | c |\}, where 'c' indicates an additional column and | indicates a vertical line separating a column. (These vertical lines can be customized to your needs.)
    \item Insert the main body of the table by entering the text you want to include in each cell and then separated each cell by \&. Insert new lines to go to the next row.
    \item Horizontal lines can be inserted within your table with \textbackslash hline.
    \item Multiple rows can be combined using \textbackslash multirow\{\textit{number\_of\_rows}\}\{width\_of\_cell (e.g., 4em. Increase number to increase width)\}\{content\_of\_cell\}. Make sure to include \textbackslash usepackage\{multirow\}.
    \item The same can be done with combining columns. Just use \textbackslash multicol and \textbackslash usepackage\{multicol\}.
\end{enumerate}
\par For example:
\newpage
\textbackslash begin\{tabular\}\{ |c|c|c| \} \\
\textbackslash hline \\
 cell1 \& cell2 \& cell3 \textbackslash\textbackslash \\
 \textbackslash hline \\
 
 \textbackslash multirow\{2\}\{4em\}\{multiple rows\} \& cell5 \& cell6 \textbackslash\textbackslash \\
 \& cell8 \& cell9 \textbackslash\textbackslash \\
 \textbackslash hline \\
\textbackslash end\{tabular\} \\
\par Creates: \\ \\
\begin{tabular}{ |c|c|c| } 
 \hline
 cell1 & cell2 & cell3 \\ 
 \hline
 \multirow{2}{4em}{multiple rows} & cell5 & cell6 \\
 & cell8 & cell9 \\
 \hline
\end{tabular}

\section*{Columns}
\begin{itemize}
    \item Include \textbackslash usepackage\{multicol\}
    \item In the beginning of your file, specify the separation of columns with \textbackslash setlength\{\textbackslash columnsep\}\{1cm\}
    \item Insert columns within your document with \textbackslash begin\{multicols\}\{\textit{number\_of\_cols}\[\textit{your\_text\_here}\]
    \item End the columns with \textbackslash end\{multicols\}.
\end{itemize}
\par See an example here: \\
\url{https://www.overleaf.com/docs?engine=pdflatex&snip_name=multicols+package+demo&snip=%5Cdocumentclass%7Barticle%7D%0A%5Cusepackage%7Bblindtext%7D%0A%5Cusepackage%7Bmulticol%7D%0A%5Csetlength%7B%5Ccolumnsep%7D%7B1cm%7D%0A%5Ctitle%7BSecond+multicols+Demo%7D%0A%5Cauthor%7BOverleaf%7D%0A%5Cdate%7BApril+2021%7D%0A%0A%5Cbegin%7Bdocument%7D%0A%5Cmaketitle%0A%0A%5Cbegin%7Bmulticols%7D%7B2%7D%0A%5B%0A%5Csection%7BFirst+Section%7D%0AAll+human+things+are+subject+to+decay.+And+when+fate+summons%2C+Monarchs+must+obey.%0A%5D%0A%5Cblindtext%5Cblindtext%0A%5Cend%7Bmulticols%7D%0A%0A%5Cend%7Bdocument%7D}

\section*{Miscellaneous}
\begin{itemize}
    \item Insert hyperlinks by including \textbackslash usepackage\{hyperref\} at the beginning of your file. Then use \textbackslash url\{\textit{your\_url}\} to insert a hyperlink in your document.
    \item Text can be colored with \textbackslash usepackage{xcolor}. Then use \textbackslash color\{\textit{your\_color}\} to specify a different color. See \url{https://www.overleaf.com/learn/latex/Using_colors_in_LaTeX} for colors.
\end{itemize}

\end{document}
